\chapter{Définitions}
\section{Avancement}
\begin{de}
 Considérons la réaction suivant :
$$\nu_1A_1 + \nu_2A_2 + .... \rightarrow \nu_1'A_1' + \nu_2'A_2' + .....$$
L'avancement à l'instant t est :
$$\xi(t) = \dfrac{n_i - n_i(t)}{\nu_i} = \dfrac{n_i'(t) - n_i'}{\nu_i'} $$
\end{de}
L'avancement ne dépend pas des entités chimique mise en jeux. C'est une grandeur caractéristique de la réaction.
\section{Application du Ier principe}
\subsection{Résultat calorimétrique}
\begin{itemize}
 \item[$\rightarrow$] Si un système subit une évolution isochore, nous savons que : $$\Delta u = Q_v$$
 \item[$\rightarrow$] Si un système subit une évolution mono ou isobare, alors :
$$\Delta H = Q_p$$ Ceci n'est verifié, pour une évolution monobar, si $p_i = p_f$, ce qui est généralement le cas en chimie
\end{itemize}
\section{Grandeur chimique de réaction}
Soit X une grandeur caractéristique du système. $$X(T,p,...n_i...)$$ avec $n_i(t)$ nombre de moles de réactif $A_i$ ou de produit $A_i'$ à l'instant t.\
\begin{de}
On appelle $X_m$ de réaction l'expression $\Delta_r X$ défini par :
$$\Delta_r X = \sum_{i=1}^N \nu_i X_{m,i}$$
On l'appelle opérateur de Lewis.
\end{de}
En t et t+dt, a T et p fixé, on obtient : 
$$dX(...n_i...) = \Delta_r X d\xi$$
On obtient aussi la relation :
$$\left(\dfrac{dX}{d\xi}\right)_{T,p}=\Delta_r X$$
\section{Grandeur standard de réaction}
\subsection{Conditions standard}
\begin{de}
 On considère qu'un système est dans les conditions standard si et seulement si :
$$p = p^0 = 1 bar$$
Avec une temperateur quelconque.
En géneral, dans les grandeur tabulées, T = 298 K.
\end{de}
\subsection{Etat standard d'une entité chimique}
\begin{de}
 On appelle état standard d'une entité chimique son état physique de référence dans les conditions standard. Ces états sont dépendants de la température.
\end{de}
\subsection{Grandeur standard de réaction}
\begin{de}
 On appelle grandeur standard de réaction une grandeur de réaction prise dans les conditions standard. On la note :
$$\Delta_r X^0 = \sum_{i=1}^N \nu i X_{m,i}^0$$
\end{de}
\subsection{Transfert thermique associé à une réaction chimique}
Dans le cas d'une évolution isochore, mono ou isotherme, on a :
$$Q_v = \Delta u^0 (T) = \Delta_r u^0(T) \xi(T)$$
Dans le cas d'une évolution mono ou isobare, mono ou isotherme, on a :
$$Q_p = \Delta H^0 (T) = \Delta_r H^0(T) \xi(T)$$
\chapter{Enthalpie standard de formation}
\begin{de}
On appelle enthalpie standard de formation d'une entité chimique, notée $\Delta_f H^o$, l'enthalpie standard de réaction de la formation de l'entité, à partir des corps simple qui la compose, dans les conditions standard, à une température T donnée.\\
Il en découle donc que l'enthalpie standard de formation d'un corps simple est nul.
\end{de}
\section{Loi de Hess}
D'après la définition ci-dessus, on obtient la loi suivante, dit loi de Hess, pour la grandeur X :
$$\Delta_r X = (\sum_i \nu_i X_{i_f} (produit_i) - \sum_i \nu_i X_{i_f} (reactif_i))$$
\section{Influence de la température sur l'enthalpie standard de réaction}
Considérons la réaction :
$$\nu_1A_1 + \nu_2A_2 + .... \rightarrow \nu_1'A_1' + \nu_2'A_2' + .....$$
On obtient :
$$\Delta_r H^0(T_2) = \Delta_r H^0(T_1) + \int_{T_1}^{T_2} (\Delta_r C_{p,mol}^0dT)$$
\section{Température de flamme}
Considérons un système qui subit une évolution adiabatique, durant la quel une réaction chimique dégage une transfert thermique Q.\\
Notons $n_i$ la quantité de matière de l'entité i, à l'équilibre.\\
On obtient la relation suivante, qui permet de déterminer la température de flamme : 
$$Q + \int_{T_{ini}}^{T_{fla}} \sum_i n_i.Cp_{mol,i}^0 dT = 0$$
On obtient la température finale des entités présentes en fin de réaction, appelé temperature de flamme.
\chapter{Énergie de liaison}
Considérons une molécule A-B à l'etat gazeuse.
\begin{de}
 L'énergie de la liaison A-B est l'énergie qu'il faut fournir à cette molécule pour rompre la liaison et produire les entités A et B à l'état gazeux.
$$A-B_{(g)} \rightarrow A_{(g)}^0 + B_{(g)}^0$$
On la note :
$$E_{Liaison}(A-B)=D(A-B)$$
\end{de}
D'un point de vu thermodynamique : 
$$D(A-B)=\Delta_r H^0 (T)$$
\chapter{Énergie d'ionisation et d'attachement électronique}
\begin{de}
 L'énergie de ionisation d'un atome X donnée est l'énergie qu'il faut fournir pour arracher un électron à l'atome $X_{(g)}$ au cours de la réaction :
$$X_{(g)} \rightarrow X^+_{(g)}+1e^-_{(g)}$$
L'énergie d'ionisation est l'enthalpie standard de cette réaction
\end{de}
Pour T $\rightarrow$ 0K, ce modèle concordre avec celui de la mécanique quantique.
\begin{de}
 L'énergie d'attachement électronique,notée $E_{AE}(T)$, est l'opposé de $\Delta_r H^0$ de la réaction suivante :
$$X_{(g)} + 1e_{(g)}^- \rightarrow X_{(g)}^-$$
$$E_{AE}(T) = -\Delta_r H^0(T)$$
\end{de}
