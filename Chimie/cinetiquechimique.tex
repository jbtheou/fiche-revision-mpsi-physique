%%%% Relecture : 20 novembre 2008
\chapter{Cinétique Chimique}
\section{Avancement - Vitesse d'une réaction chimique}
Considérons la réaction suivant :
$$\nu_1.A_1 + \nu_2.A_2 + .... \rightarrow \nu_1'.A_1' + \nu_2'.A_2' + .....$$
\subsection{Avancement d'une réaction chimique}
\begin{de}
L'avancement à l'instant t est donnée par :
$$\xi(t) = \dfrac{n_i - n_i(t)}{\nu_i} = \dfrac{n_i'(t) - n_i'}{\nu_i'} $$
On obtient donc les relations, en utilisant les concentrations au lieu des quantités de matières :
$$[A_i(t)] = [A_i(0)] - \nu_i.x(t)$$
$$[A'_i(t)] = [A'_i(0)] + \nu'_i.x(t)$$
\end{de}
\subsection{Vitesse d'une réaction chimique}
\begin{de} 
La vitesse de réaction est définie par :
$$v = \dfrac{1}{\nu'_1}.\dfrac{d[A'_i(t)]}{dt} = -\dfrac{1}{\nu_1}.\dfrac{d[A_i(t)]}{dt}$$
D'où, d'après les relations précédents :
$$v(t) = \dfrac{dx(t)}{dt}$$
\end{de}
\subsubsection{Facteur cinétique}
Il existe certains facteurs pouvant influer sur la vitesse d'une réaction :
\begin{itemize}
 \item[$\rightarrow$] La température
 \item[$\rightarrow$] La concentration
 \item[$\rightarrow$] L'influence éventuelle d'un catalyseur
 \item[$\rightarrow$] La pression exterieur, dans le cas d'une réaction entrainant un dégagement gazeux
 \item[$\rightarrow$] L'éclairage de la solution
 \item[$\rightarrow$] La surface de contact en phase hétérogène (ex : corrosion d'un bateau)
\end{itemize}
\subsubsection{Influence de la concentration}
\begin{de}
Considérons la réaction suivant :
$$\nu_1.A_1 + \nu_2.A_2 + .... \rightarrow \nu_1'.A_1' + \nu_2'.A_2' + .....$$
On dit que cette réaction admet un ordre si elle peut s'exprimer sous la forme, si la température de la solution est constante :
$$v = k[A_1(t)]^{\alpha_1}.k[A_2(t)]^{\alpha_2}.....$$ 
avec : $\alpha_i$ : coefficiant déterminé par l'expérience. On détermine l'ordre global de cette réaction par :
$$\alpha = \sum_i \alpha_i$$
\end{de}
\subsubsection{Influence de la température}
En faisant l'hypothèse que la réaction précedente admet un ordre, on montre que la constante k dépend de la température. On obtient la loi d'Arrhenius :
$$k = Ae^{-\dfrac{E_{m,A}}{RT}}$$
avec A le facteur d'Arrhenius, défini par l'expérience et $E_{m,A}$ l'énergie d'activation molaire.
\section{Étude de cinétique}
\subsection{Méthode de séparation d'Ostwald}
Cette méthode consiste à augmenter les concentrations de tous les réactifs, sauf un, pour pouvoir étudier celui. Le rapport des concentrations doit être de 100 au moins. On néglige donc l'influence de tous les autres réactifs, on obtient donc une équation à un seul réactif, mais on conserve les n produits.
\subsection{Reconnaisance des ordres par rapport à [A(t)]}
Le but est d'obtenir comme courbe une fonction affine, pour pouvoir déterminer k facilement.\\
Pour obtenir la fonction affine, on utilise la définition de la vitesse :
$$v = \dfrac{1}{\nu'_1}.\dfrac{d[A'_i(t)]}{dt} = -\dfrac{1}{\nu_1}.\dfrac{d[A_i(t)]}{dt} = \dfrac{dx(t)}{dt}$$
Et on fait le lien avec l'expression en fonction de k et des concentrations.\\
Soit a la concentration initiale.
\begin{itemize}
 \item[$\rightarrow$] Ordre 0 : On obtient une droite, de pente $-\nu k$ avec : $$y = [A(t)]$$
 \item[$\rightarrow$] Ordre 1 : On obtient une droite, de pente $-\nu k$ avec : $$y = ln\left( \dfrac{a}{[A(t)]}\right) $$
 \item[$\rightarrow$] Ordre 2 : On obtient une droite, de pente $\nu k$ avec : $$y = \left( \frac{1}{[A(t)]} - \frac{1}{a}\right) $$
\end{itemize}
\subsection{Réaction composée}
Dans l'étude des réactions composée, on procède de la même façon que précédement. En considérant soit, qu'il s'effectue deux réactions opposées en parralèle, soit qu'on est dans le cas de réactions successives.
