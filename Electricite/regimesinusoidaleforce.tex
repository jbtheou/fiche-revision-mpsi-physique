\chapter{Régime sinusoidale forcé}
\section{Réponse d'un circuit R.L.C série à une excitation sinusoïdale}
Considérons une association R.L.C., contenant un GBF.\\
Le GBF fourni une tension sinusoïdale e(t) :
$$e(t) = Ecos(\omega t +\varphi)$$
avec $\omega = 2\pi.f$.\\
À l'aide de la loi des mailles, on obtient :
$$\dfrac{d^2u_c(t)}{d^2t} + \dfrac{w_0}{Q}\dfrac{du_c(t)}{dt} + \omega_0^2u_c(t) = \omega_0^2e(t)$$
\subsection{Réponse de $u_c(t)$}
La solution de cette équation est de la forme :
$$u_c(t) = u_{c1}(t) + u_{c2}(t)$$
Avec : 
\begin{itemize}
 \item[$\rightarrow$] $u_{c1}(t)$ : Solution de l'équation homogène, caratéristique d'un régime transitoire.
 \item[$\rightarrow$] $u_{c2}(t)$ : Solution particulière de l'équation avec second membre, caratéristique d'un régime permanent :
$$u_{c2}(t) = u_ccos(\omega t+\psi)$$
\end{itemize}
\section{Impedance complexe}
\subsection{Impedance complexe d'un dipole}
Considérons un dipole en régime sinusoïdale forcé.\\
On obtient :
$$\underline{i}(t) = \underline{I_M}e^{j\omega t}$$
avec $\underline{I_M} = I_Me^{j\varphi}$
$$\underline{u}(t) = \underline{U_M}e^{j\omega t}$$
avec $\underline{U_M} = U_Me^{j\psi}$\\
\begin{de}
On appele impédance complexe le scalaire défini par :
$$\underline{u}(t) = \underline{Z}.\underline{i}(t)$$
D'où :
$$\underline{Z} = \dfrac{\underline{U_M}}{\underline{I_M}}$$
Soit $\alpha$ l'argument de $\underline{Z}$ :
$$\alpha = \psi - \varphi$$
On observe que : 
\begin{itemize}
 \item[$\rightarrow$] $\alpha = 0$ : u(t) et i(t) sont en phase
 \item[$\rightarrow$] $0 < \alpha < \pi$ : u(t) est en avance par rapport à i(t) dans le dipole
 \item[$\rightarrow$] $-\pi < \alpha < 0$ : u(t) est en retard par rapport à i(t) dans le dipole
\end{itemize}
\end{de}
\subsubsection{Impédance d'un conducteur ohmique}
Pour un conducteur ohmique, on obtient :
$$\underline{Z_R} = R$$
On observe donc que $\alpha = 0$, il n'y a donc pas de déphasage entre u(t) et i(t) dans une résistance.
\subsubsection{Impedance d'un solénoïde}
Pour un solénoïde, on obtient :
$$\underline{Z_L} = j\omega L$$
On observe donc que $\alpha = \dfrac{\pi}{2}$ : u(t) est en quadrature avance par rapport à i(t)
\subsubsection{Impedance d'un condensateur}
Pour un condensateur, on obtient :
$$\underline{Z_C} = \dfrac{1}{j\omega C}$$
On observe donc que $\alpha = \dfrac{-\pi}{2}$ : u(t) est en quadrature retard par rapport à i(t)
\subsubsection{Loi d'additivité}
\begin{itemize}
 \item[$\rightarrow$] En série : $$\underline{Z_{eq}} = \sum_{k=1}^n \underline{Z_k}$$
 \item[$\rightarrow$] En dérivation : $$\underline{\dfrac{1}{Z_{eq}}} = \sum_{k=1}^n \dfrac{1}{\underline{Z_k}}$$
\end{itemize}
\section{Diagramme de Fresnel}
Considérons une association R.L.C. en série.\\
Par application de la loi des mailles :
$$\underline{E_M} = \underline{U_R} + \underline{U_C} +\underline{U_L}$$
On construit à partir de ce moment le diagramme de Fresnel :
\begin{enumerate}[1 -]
 \item On défini une orientation pour les angles, puis on fixe un axe de référence, en général $\underline{U_R} = R.\underline{I_M}$.\\
 \item Puis on ajoute $\underline{U_C}$ et $\underline{U_L}$ en respectant la convention d'orientation des angles.
\end{enumerate}
Grâce à ce diagramme, on détermine rapidement le déphase $\alpha$ entre $\underline{I_M}$ et $\underline{E_M}$ à l'aide de tan($\alpha$) :
$$tan(\alpha) = \dfrac{\underline{Z_L}-\underline{Z_C}}{\underline{Z_R}}$$
Dans le cas d'un circuit en dérivation, on peut définir ce diagramme à l'aide des courants et de la loi des noeuds.
\section{Théorèmes généraux en régime sinusoidale forcé}
\subsection{Lois de Kirchhoff}
\subsubsection{Loi des noeuds}
$$\sum_{Entrant}\underline{I_M} = \sum_{Sortant}\underline{I_M}$$
\subsubsection{Loi des mailles}
$$\sum_{k=1}^n\underline{U_{M_k}} = 0$$
\subsection{Diviseur de tension}
Considérons l'association de n conducteurs ohmiques en série entre deux points A et B.\\
Soit $U_j$ la tension parcourant le $j^{eme}$ conduteur ohmique, de résitance $R_j$ :
$$\underline{U_{M_j}} = \left( \dfrac{\underline{Z_j}}{\sum_{k=1}^n \underline{Z_k}}\right)\underline{U_{M}} $$
\subsection{Diviseur de courant}
Considérons deux résitances en dérivation.\\
Soit i le courant entrant, $i_1$ le courant traversant la résitance $R_1$.\\
On obtient :
$$\underline{I_{M_1}} = \left( \dfrac{\underline{Z_2}}{\underline{Z_1}+\underline{Z_2}}\right)\underline{I}$$
\subsection{Théorème de Kennely}
Considérons un montage en étoile contenant les dipoles d'impédances complexes $\underline{Z_1},\underline{Z_2},\underline{Z_3}$.\\
On obtient le montage équivalent, avec un seul noeud, contenant les dipoles d'impédances complexes $\underline{Z'_1},\underline{Z'_2},\underline{Z'_3}$, à l'aide des relations :
$$\underline{Z'_1} = \dfrac{\underline{Z_2}\underline{Z_3}}{\sum_i \underline{Z_i}}$$
avec $\underline{Z'_1}$ du coté de A, on a donc au numérateur $\underline{Z_2}\underline{Z_3}$, qui sont eux aussi du coté de A.
\subsection{Théorème de Millman}
Considérons un circuit composé de plusieurs dipôles : Sources de courants, sources de tensions, résistances.
Soit A et B deux points du circuits.\\
Le théorème de Millman permet de déterminer la tension $U_{AB}$ de la façon suivante :
$$\underline{E_M}_{AB} = \dfrac{\sum_{j=1}^m \dfrac{\underline{E_M}_{jB}}{\underline{Z_j}} + \sum_{k=1}^M \underline{I_M}_k}{\sum_{j=1}^n \dfrac{1}{\underline{Z_j}}}$$
\section{Puissance instantanée et puissance moyenne}
\subsection{Puissance instantanée}
Considérons un dipole quelconque en convention récepteur.
\begin{de}
On appele puissance instantanée, notée p(t), dans le dipole AB, en convention récepteur, le produit :
$$p(t)=u(t).i(t)$$
Son unité est le Watt
On obtient : 
$$p(t) = U_M.cos(\omega t+\varphi).I_M.cos(\omega t)$$
p(t) ne s'exprime jamais avec des grandeurs complexes.
\end{de}
\subsection{Puissance moyenne}
\begin{de}
La puissance moyenne, notée P(t), d'un signal periodique de periode T est défini par :
$$P(t) = \dfrac{1}{T}\int_t^{t+T}p(t)dt$$
On obtient que la puissance moyenne consommée par un dipole en régime sinusoidale forcé est :
$$P(t) = \dfrac{U_M.I_M}{2}cos(\varphi)$$
avec $\varphi$ le déphasage entre u(t) et i(t) dans le dipole 
\end{de}
\subsection{Grandeur efficace}
\begin{de}
Soit f(t) une fonction periodique, de periode T.\\
La valeur efficace de f(t), notée $F_e$, est défini par :
$$F_e = \sqrt{<f(t)^2>}$$
avec $<f(t)^2>$ la valeur moyenne de $f(t)^2$
\end{de}
En régime sinusoïdale forcé, on obtient :
$$U_e = \dfrac{U_M}{\sqrt{2}}$$
$$I_e = \dfrac{I_M}{\sqrt{2}}$$
\subsection{Facteur de puissance}
\begin{de}
Avec les valeurs efficaces en régime sinusoïdale forcé, on obtient :
$$P(t) = U_e.I_e.cos(\varphi)$$
On appele $cos(\varphi)$ facteur de puissance du dipole
\end{de}
\subsection{Puissance consommée, Puissance dissipé}
En régime sinusoïdale forcé, la puissance moyenne consommée par une résistance est :
$$P_R(t) = \dfrac{R.I_M^2}{2}$$
La puissance moyenne consommée par un solénoïde ou par un condensateur est nul car $\varphi = \pm \dfrac{\pi}{2}$ 
