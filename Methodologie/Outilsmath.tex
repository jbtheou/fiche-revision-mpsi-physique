\chapter{Outils Math\'ematique}
\section{Produit scalaire}
Soit $\overrightarrow{X}(x_1,y_1,z_1),\overrightarrow{Y}(x_2,y_2,z_2)$.\\
Le produit scalaire est défini par :
$$\overrightarrow{X}.\overrightarrow{Y}=x_1x_2+y_1y_2+z_1z_2$$
$$\overrightarrow{X}.\overrightarrow{Y}=||\overrightarrow{X}||.||\overrightarrow{Y}||.cos(\theta)$$
\section{Produit vectorielle}
$$\overrightarrow{X}\wedge\overrightarrow{Y} = \begin{pmatrix}
  x_1 \\
  y_1 \\
  z_1 \\
\end{pmatrix}\wedge\begin{pmatrix}
  x_2 \\
  y_2 \\
  z_2 \\
\end{pmatrix} = \begin{pmatrix}
  y_1z_2 - z_1y_2 \\
  z_1x_2 - x_1z_2 \\
  x_1y_2 - y_1x_2 \\
\end{pmatrix}$$
$$\overrightarrow{X}\wedge\overrightarrow{Y}=\overrightarrow{W}$$
avec $\overrightarrow{W}$ orthogonale à $\overrightarrow{X}$ et $\overrightarrow{Y}$
\section{Dérivation d'un vecteur}
$$\dfrac{d\overrightarrow{X}}{dt} = \overrightarrow{W}\wedge\overrightarrow{X}$$
avec $\overrightarrow{W}$ le vecteur rotation instantanée.\\
Appliqué à $\overrightarrow{u_{\theta}}$ et $\overrightarrow{u_{r}}$, on obtient :
$$\dfrac{d\overrightarrow{u_{\theta}}}{dt} = -\mathring{\theta}\overrightarrow{u_{r}}$$
$$\dfrac{d\overrightarrow{u_{r}}}{dt} = \mathring{\theta}\overrightarrow{u_{\theta}}$$
\section{Coordonnée}

\subsection{Coordonnée polaire}
$$\overrightarrow{dl} = dr.\overrightarrow{u_r}+rd\theta.\overrightarrow{u_{\theta}}$$
\subsection{Coordonnée cylindrique}
$$\overrightarrow{dl} = dr.\overrightarrow{u_r}+rd\theta.\overrightarrow{u_{\theta}}+dz.\overrightarrow{k}$$
\subsection{Coordonnée sphérique}
$$\overrightarrow{dl} = dr.\overrightarrow{u_r}+rd\theta.\overrightarrow{u_{\theta}}+rsin(\theta)d\varphi.\overrightarrow{u_{\varphi}}$$
\section{Equations différentielles}
\begin{itemize}
 \item[$\rightarrow$] $x'+\alpha x=0$ $$x(t) = Ae^{-\alpha t}$$
 \item[$\rightarrow$] $x'' + \omega_0^2x = 0$ $$x(t) = C.cos(\omega_0t+\phi)$$
 \item[$\rightarrow$] $x'' - \omega_0^2x = 0$ $$x(t) =  Ach(\omega_0t)+Bsh(\omega_0t)$$
 \item[$\rightarrow$] $x'' + 2\alpha x'+\omega_0^2x = 0$ 
\begin{itemize}
 \item[$\rightarrow$] Si $\Delta$ > 0 : $$x(t) = Ae^{r_1t} + Be^{r_2t}$$
 \item[$\rightarrow$] Si $\Delta$ < 0 : $$x(t) = Ce^{-\alpha t}cos(\omega t+\phi)$$ 
 \item[$\rightarrow$] Si $\Delta$ = 0 : $$x(t) = Ae^{- \alpha t}(A+Bt)$$
\end{itemize}
\end{itemize}
