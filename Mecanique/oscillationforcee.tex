\chapter{Oscillation Forcée}
\section{Pendule horizontale soumis à une excitation sinusoïdale}
Considérons un mobile M(m) de masse m, observé dans le référentiel terrestre R galiléen. M est soumis à :
\begin{itemize}
 \item[$\rightarrow$] Son poid $\overrightarrow{p}$
 \item[$\rightarrow$] La réaction normale au support $\overrightarrow{n}$
 \item[$\rightarrow$] Une force de frottement fluide : $\overrightarrow{f}=-\alpha\overrightarrow{v}(M)_R$
 \item[$\rightarrow$] L'action du ressort : $\overrightarrow{T} = -k\Delta l$
\end{itemize}
L'excitateur est fixé au ressort. Le point de fixation peut se déplacer par rapport à sa position centrale $O_1$.\\
Soit $\overrightarrow{OE_1}(t) = x_e(t)\overrightarrow{i}$ l'écartement du point de fixation par rapport à sa position centrale.\\
L'excitateur impose une excitation de type sinusoïdale :
$$x_e(t) = X_E.cos(\omega t)$$
En l'absence d'excitation, quand M est au repos, la position de M est repèré par le point O.\\
On appelera $x(t)$ l'ecartement de M par rapport à O : 
$$\overrightarrow{OM(t)} = x(t).\overrightarrow{i}$$
\subsection{Équation différentielle du mouvement}
En appliquant le P.F.D à M(m) dans R galiléen, on obtient : 
$$\mathring{x}\mathring{} + \dfrac{\alpha}{m}\mathring{x} +x(t)\dfrac{k}{m} =x_e(t)\dfrac{k}{m}$$
En posant : 
\begin{itemize}
 \item[$\rightarrow$] $\omega_0 = \sqrt{\dfrac{k}{m}}$\\
 \item[$\rightarrow$] $\dfrac{\alpha}{m} = \dfrac{\omega_0}{Q}$
\end{itemize}
On obtient :
$$\mathring{x}\mathring{} + \dfrac{\omega_0}{Q}\mathring{x} +\omega_0^2x(t) =\omega_0^2X_Ecos(\omega t)$$
\subsection{Régime forcé ou régime permanent}
\begin{de}
On considère que le système fonctionne en régime permanent quand $x(t)$ est à peu près égale à la solution particulière de l'équation différentielle, à savoir quand :
$$x(t) \simeq X_Ecos(\omega t+\varphi)$$
On appelle X l'amplitude de l'élongation de l'oscillateur et $\varphi$ le déphasage entre l'oscilatteur et l'excitateur.
\end{de}
\section{Résonance en élongation}
\subsection{Notation complexe}
Pour étudier le système, nous allons définir les notations complexes :
$$x_e(t) = X_E cos(\omega t) \leftrightarrow \underline{x_e}(t) = X_E e^{i\omega t}$$
$$x(t) = X cos(\omega t + \varphi) \leftrightarrow \underline{x}(t) = \underline{X}e^{i\omega t}$$
avec $\underline{X} = Xe^{i\varphi}$ l'amplitude complexe de l'oscillateur
\subsection{Amplitude complexe de l'oscillateur}
En injectant dans l'équation différentielle, sachant que si par exemple, A est un complexe, on obtient :
$$\dfrac{d}{dt}(A) = i\omega A$$
On obtient donc :
$$\underline{X}(u) = \dfrac{X_E}{1-u^2+i\dfrac{u}{Q}}$$
avec u, pulsation réduite :
$$u = \dfrac{\omega}{\omega_0}$$
On peut donc déterminer le module de l'amplitude de l'oscillateur et la phase $\varphi$ :
$$X(u) = \dfrac{X_E}{\sqrt{(1-u^2)^2 + \dfrac{u^2}{Q^2}}}$$
$$tan(\varphi) = \dfrac{u}{Q(1-u^2)}$$
\subsection{Résonance d'élongation}
Par analogie avec l'électricité, il y a résonance d'élongation si et seulement si :
$$Q > \dfrac{1}{\sqrt{2}} \simeq 0,7$$
À la résonance d'élongation, on obtient les relations suivantes :
$$u_{r} = \sqrt{1 - \dfrac{1}{2Q^2}} \simeq 1 \mbox{ Pour Q grand}$$
$$X_{max} = X(u_r) = \dfrac{QX_E}{\sqrt{1 - \dfrac{1}{4Q^2}}} \simeq QX_E \mbox{ Pour Q grand}$$
Pour Q grand, l'oscillateur est en quadrature retard par rapport à l'excitateur, d'apres le profil de tan($\varphi$) 
\section{Vitesse}
\subsection{Amplitude complexe de la vitesse}
\begin{de}
L'amplitude complexe de la vitesse est : 
$$\underline{v}(t) = i\omega \underline{x}(t)$$
En posant : 
\begin{itemize}
 \item[$\rightarrow$] $ \underline{v}(t) = \underline{V}(\omega)e^{i\omega t}$
 \item[$\rightarrow$] $ \underline{x}(t) = \underline{X}(\omega)e^{i\omega t}$
\end{itemize}
On obtient : 
$$\underline{V}(\omega) = i\omega\underline{X}(\omega)$$
En posant : 
$$\underline{V}(\omega) = V(\omega)e^{i\psi(\omega)}$$
avec :
\begin{itemize}
 \item[$\rightarrow$] $V(\omega)$ : module de la vitesse complexe
 \item[$\rightarrow$] $\psi(\omega)$ : déphasage de la vitesse par rapport à l'exciateur
\end{itemize}
\end{de}
Conscient que : \\
\begin{itemize}
 \item[$\rightarrow$]  $\underline{X}(\omega) = X(\omega)e^{i\varphi(\omega)}$\\
 \item[$\rightarrow$] $i = e^{i\dfrac{\pi}{2}}$\\
\end{itemize}
On obtient : 
$$\begin{cases}
   V(\omega) = \omega.X(\omega) = \dfrac{\omega_0.u.X_e}{\sqrt{(1-u^2)^2 + \dfrac{u^2}{Q^2}}}\\
   \psi(\omega) = \varphi(\omega) + \dfrac{\pi}{2}\\
  \end{cases}$$
On montre que $\forall Q$, pour $u_r = 1$, il y a résonance de vitesse.\\
À la résonance de vitesse : 
\begin{itemize}
\item[$\rightarrow$] $V_{max} = V(\omega_0) = Q\omega_0X_e$
\item[$\rightarrow$] $\psi(\omega_0) = 0$ : La vitesse est en phase avec l'excitateur.
\end{itemize}
\section{Impedence complexe}
\begin{de}
Considérons un oscillateur mécanique soumis à l'action d'une force excitatrice d'amplitude complexe $\underline{F}$.\\
On défini l'impedence complexe d'un oscillateur mécanique par le rapport :
$$\underline{Z} = \dfrac{\underline{F}}{\underline{V}}$$
avec $\underline{V}$ la vitesse complexe
\end{de}
\subsection{Force explicite}
D'apres l'équation différentielle du mouvement de M(m), on obtient l'impedence complexe d'une oscillateur mécanique :
$$\underline{Z}(\omega) = im\omega + \alpha + \dfrac{k}{i\omega}$$
\subsection{Analogie électro-mécanique}
On observe l'équivalence formel des grandeurs suivantes :\\
\begin{itemize}
 \item[$.$] Électricité $\rightarrow$ Mécanique\\
  \item[$.$] q(t) $\rightarrow$ x(t)\\
 \item[$.$] i(t) $\rightarrow$ v(t)\\
 \item[$.$] u(t) $\rightarrow$ f(t)\\
 \item[$.$] R $\rightarrow$ $\alpha$\\
 \item[$.$] L $\rightarrow$ m\\
 \item[$.$] $\dfrac{1}{c}$ $\rightarrow$ k\\
\end{itemize}
