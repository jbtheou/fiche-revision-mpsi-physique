\chapter{\'Energie d'un point materiel}
\section{Puissance et travail d'une force}
\subsection{Puissance}
\begin{de}
Soit M(m) un point materiel de masse m observé dans un référentiel R galiléen et animé de la vitesse $\overrightarrow{V}(M)_R$.\\
Supposons que M soit soumis à l'action d'une force $\overrightarrow{F}$.\\
La puissance, scalaire associée à cette force à chaque instant, est définie par :
$$P = \overrightarrow{F}.\overrightarrow{V}(M)_R$$
L'unité de la puissance est le Watt.
\end{de}
\begin{prop}
Si M(m) est soumis à un ensemble de forces $\Sigma \overrightarrow{F}$, alors :
$$P = \sum_i P_i$$
avec :
$$P_i = \overrightarrow{F_i}.\overrightarrow{V}(M)_R$$
\end{prop}
\subsection{Travail d'une force}
\begin{de}
Le travail d'une force $\overrightarrow{F}$ entre t et t+dt dans le référentiel R est donné par :
$$\delta\omega = P.dt$$
On obtient aussi la formule :
$$\delta\omega = \overrightarrow{F}.\overrightarrow{dl}$$
\end{de}
\section{Théorème de l'énergie cinétique}
\subsection{Énergie cinétique}
\begin{de}
Soit M(m) point materiel de masse m observé dans R galiléen, et animé de la vitesse $\overrightarrow{V}(M)_R$.\\
L'énérgie cinétique de M(m) dans R est donnée par, avec v la norme de $\overrightarrow{V}(M)_R$:
$$E_c(M)_R = \dfrac{1}{2}.m.v^2(M)_R$$
L'unité de l'énergie cinétique est le Joule.
\end{de}
\subsection{Théorème de l'énergie cinétique}
Soit M(m) un point matériel de masse m observé dans R galiléen, soumis dans R à la résultante $\Sigma \overrightarrow{F}$ des forces exterieures.\\
\begin{theo}
On obtient la relation :
$$dE_c(M)_R = \sum_i \delta\omega_i$$
\end{theo}
Entre deux instants $t_1$ et $t_2$, le théorème de l'énergie cinétique devient :
$$\Delta_{t_1 \rightarrow t_2} E_c(M) = \sum_i \omega_i$$
avec :
$$\omega_{t_1 \rightarrow t_2} = \int_{M_1}^{M_2} \overrightarrow{F}.\overrightarrow{dl}$$
\section{Force conservative}
\begin{de}
Une force $\overrightarrow{F}$ est dite conservative quand son travail entre deux points $M_1$ et $M_2$ ne dépend pas du chemin suivit, mais uniquement des positions $M_1$ et $M_2$.\\
Ceci implique que sur un contour fermé, le travail d'une force conservative est nul.\\
Une force qui n'est pas conservative est dites non-conservative.
\end{de}
\subsection{Force conservative et énergie potentielle}
On obtient la formule suivante, liant énergie potentielle et force conservative : 
$$\delta \omega = \overrightarrow{F}.\overrightarrow{dl}=-dE_p$$
\section{\'Energie mécanique et intégrale première de l'énergie cinétique}
\subsection{Energie mécanique}
\begin{de}
L'énergie mécanique d'un point materiel M(m) de masse m dans un réferentiel R, notée $E_m(M)_R$, est la somme de son énergie cinétique et de son énergie potentielle totale $E_{p_{tot}}$ :
$$E_m(M)_R = E_c(M)_R + E_{p_{tot}}$$
\end{de}
\subsection{Relation entre le théorème de l'énergie cinétique et $E_m(M)_R$}
Soit M(m) un point materiel de masse m, soumis à des forces non-conservatives, notées $F^{n-c}$.
On obtient la relation suivant :
$$dE_m(M)_R = \Sigma \delta \omega^{n-c}$$
\subsection{Intégrale première de l'énergie cinétique}
L'énergie mécanique de M dans R est constante si et seulement si il n'y a pas de forces non conservatives ou les forces non conservatives ne travaillent pas, c'est à dire que : $$\overrightarrow{F}^{n-c} \bot \overrightarrow{dl}$$
Dans ce cas, en dérivant l'expression de l'énergie mécanique, on obtient une expression égale à zéro. Ce qui permet de ne pas avoir à passer par le P.F.D. Ceci consitue l'intégrale première de l'énergie cinétique.
\subsection{Relation entre l'énergie potentielle et les relations d'équilibre}
Considérons un système dont la position ne dépend que d'une seule variable de l'espace.\\
Supposons que M(m), point materiel de masse m, soit soumis à le force résultant $\overrightarrow{F}$ conservative.\\
Les positions d'équilibre du système sont les extrémums de la fonction $E_p$.\\
Si la courbe est convexe au voisinage de l'équilibre, alors l'équilibre est stable.\\
Si la courbe est concave au voisinage de l'équilibre, alors l'équilibre est instable. \\
